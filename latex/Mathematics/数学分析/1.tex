\section{集合以及其基本运算}
\subsection{集合的概念}
{\large\textit{“我们把集合理解为由若干确定的、有充分区别的、具体或抽象的对象合并而成的一个整体”——康托尔}}\\


但是集合的概念并非完美无瑕,例如所有集合的集合本身就是一个矛盾的概念。

对于集合$M$,设记号$P(M)$表示$M$不以其本身为元素,考虑具有性质$p$的集合所组成的一类对象$K={M|P(M)}$\\
如果$K$是集合,则或者$P(K)$为真,或者$\neg P(K)$为真,然而这两者对于$P(K)$来说都不可能为真,因此$P(K)$不是一个集合。
这就是经典的罗素悖论,原因是朴素集合论的不够严谨导致形成的悖论。
这里并不会再对集合做进一步的定义,我们只需要知道在现有的公理化的集合论中,集合被定义为具有一组确定性质的数学对象。
\subsection{集合的相关运算}
由于中学时期我们对集合已经有了初步的了解,所以这里我就不再对集合的交集并集以及补集运算作进一步的解释,下面给出德摩根法则,有兴趣的同学可以自己证明
\begin{definition}{集合}{}
		\[C_M(A\cup   B)=C_MA\cap C_MB \]
        \[C_M(A\cap B)=C_MA\cup C_MB \]
	\end{definition}%定理的模板
需要额外介绍的是笛卡尔直积.对于任何两个集合$A,B$,可以做成一个新的集合——偶$\{A,B\}=\{B,A\}$,其元素是且仅是集合A和B.这个集合在$A\neq B$时,由两个元素组成,而在$A=B$的时候由一个元素组成。\\

上述新集合称为集合A和集合B的无序偶,以区别于序偶$(A,B)$,后者的元素$A$,$B$具有附加特征,从而能够区别偶$(A,B)$中第一个元素和第二个元素。按照定义,序偶等式
\[(A,B)=(C,D)\]
表示$A=C$且$B=D$,特别地,如果$A\neq B$,则$(A,B)\neq (B,A)$。

现在设$X,Y$是任意集合,集合
\[X\times Y:=\{(x,y)|x\in X,y\in Y\}\]
称为集合$X$和集合$Y$的笛卡尔积,或者直积。笛卡尔积的元素是所有可能的序偶$(x,y)$,其中$x\in X,y\in Y$。

从直积的定义和关于序偶的可以看出,一般上述说明而言,$X\times Y\neq Y\times X$,除非$X=Y$,此时$X\times X$简写为$X^2$。

在平面上众所周知的笛卡尔坐标系恰好把该平面变为两个数轴的直积,在这个熟悉的情形下笛卡尔积对因子顺序的依赖性明显表现出来。

设序偶$z=(x_1,x_2)$是集合$X_1$与$X_2$的直积$Z=X_1\times X_2$中的元素,则$x_1$称为序偶$z$的第一投影,记作$pr_1z$;$x_2$称为序偶$z$的第二投影,记作$pr_2z$。我们可以用投影映射来定义直积的两个映射


\section{函数与极限}
第一讲要讨论的是两个基本概念,分别是函数和极限,函数是微积分学的研究对象,极限是我们学习的工具。在学习过程中我们通常先学习极限以及实数的定义,再由此衍生到微积分。而事实上这一顺序与数学发展的顺序并不相同,牛顿莱布尼茨分别发明了微积分,而在后面若干年各学者的补充完善下才确定了实数与极限的概念。
\subsection{函数}
\begin{definition}{函数}{}
设$X$和$Y$是两集合,如果集合$X$的每一个元素$x$都按照某规律$f$与集合$Y$的元素$y$相对应,我们就说有一个函数,它定义于$X$并且取值于$Y$。
\end{definition}
\subsubsection{映射的简单分类}
当函数$f:X\rightarrow Y$被称为映射的时候,它在元素$x\in X$上的值$f(x)\in Y$通常称为元素$x$的像。
映射$f:X\rightarrow Y$分为以下几类:

满射(或者称为到上映射),此时$f(X)=Y$;

单射(或者称为嵌入),此时对于集合$X$的任何元素$x_1,x_2$有
\[(f(x_1)=f(x_2))\Rightarrow (x_1=x_2)\]
即不同元素有不同的像;

双射(或者称为一一映射),此时它即是单射又是满射。如果$f:X\rightarrow Y$是双射,那么自然就存在一个映射
\[f^{-1}:Y\rightarrow X\]
其定义方法如下:如果$f(x)=y$,则$f^{-1}:Y\rightarrow x$,即元素$y\in Y$相对应的是在映射$f$下以$y$为像的元素$x\in X$,因为f是满射,所以这样的元素x存在。又因为f是单射,所以该元素是唯一的。因此,我们可以定义映射$f^{-1}$,这个映射称为原映射逆映射。
\subsection{集合的势}
\begin{definition}{集合的势}{}
如果集合$X$到集合$Y$的双射存在,称集合$X$与集合$Y$等势
\end{definition}
集合X所在的类称为集合X的势或基数类,记为$cardX$,如果$X~Y$,即可以写出$cardX=cardY$。

这种结构的意义在于,它能够让我们比较集合中所含元素的数量,而不必采用数数的方式,事实上后者在一些情况是不可能完成的。

如果集合X与集合Y的某一个子集等势,我们说集合X的基数类不大于集合Y的基数类,并且记为$cardX\leq cardY$,而一个集合能够与自身的一部分等势,这是无穷集的特征,戴德金曾建议以此为无穷集的定义。因此有如果一个集合不与自己的任何一个真子集等势,我们就称其为有限集,反之则称之为无穷集。

定理.$cardX<cardP(X)$,其中$P$表示集合X中一切子集的集合,