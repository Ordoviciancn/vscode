\usepackage{ctex}
\documentclass{article}
\usepackage{graphicx} % Required for inserting images
\usepackage{amsmath} % 加载这个宏包以使用cases环境等数学相关功能

\title{微分方程}
\author{Felix}
\date{12.18}

\begin{document}

\maketitle

\section{常微分方程}
\subsection{微分方程的基本概念}
微分方程可以理解为满足一定条件的函数及其导数之间的关系的等式,其解为一系列函数,某一满足条件的函数为特解,所有满足条件的函数为通解

举例如下:
\[\frac{dy}{dx}=4x\]

并且$y=y(x)$满足条件
\[y|_{x=1}=3\]

将(1)式两边积分可得
\[y=2x^2+C\]

将条件(2)带入上式可得$3=2+C$,解得$C=1$,进而可得
\[y=2x^2+1\]

在上面这个例子中,该方程中含有未知函数的导数(或者微分),此时若未知函数是一元函数(仅包含一个未知数),则称该方程为常微分方程

特别的,我们将微分方程中出现的未知函数的最高阶导数的阶数叫做微分方程的阶

给出通解和特解的进一步定义:如果微分方程的解中含有任意常数,且任意常数的个数与微分方程的阶数相同,这样的解叫做微分方程的通解;若微分方程的解不含任意常数,这样的解叫做微分方程的特解

可以通过初始条件来却大幅微分方程的某一特解,求微分方程满足初始条件的特解,这一问题叫做微分方程的初值问题或柯西问题

\subsection{一阶微分方程}
必须要知道的是,很遗憾,我们很难解决绝大部分微分方程,能解出的微分方程必须满足一定特性或者模式,以下是一些非常基础且常用的方法
\subsubsection{分离变量法}
形如$$\frac{dy}{dx}=f(x)g(y)$$的一阶微分方程是可分离变量的微分方程,其中$f(x)$$g(x)$为已知的连续函数

当$g(y)\neq0$时,可以化简为(这里要小心丢解)
\[\frac{dy}{g(y)}=f(x)dx\]

设函数$y=y(x)$为微分方程的任一解,带回上式,得
\[\frac{y'(x)}{g(f(x))}dx=f(x)dx\] 

两边同时积分可得:
\[\int{\frac{dy}{g(y)}}=\int{f(x)dx}\]

由此可以得到微分方程的通解

分离变量法还可以在一些其它题型中有应用,例如在齐次方程中也有应用

如果一阶常微分方程能够化为形如
\[\frac{dy}{dx}=f\left(\frac{y}{x}\right)\]
的形式,那么则称其为齐次方程

通过变量代换可以将其化为可分离变量方程,令$u=\frac{x}{y}$,即$x=yu$,然后对等式两边求导,得到
\[\frac{dy}{dx}=u+y\frac{du}{dx}\]
带回微分方程可以得到可分离变量的微分方程
\subsubsection{更具有一般性的方程}
这里是形如
\[\frac{dy}{dx}=f\left(\frac{ax+by+c}{a_1x+b_1y+c_1}\right)\]
的方程称为更具有一般性情况,特别的当$c=c_1=0$时,方程本身就是齐次方程,当c和$c_1$不全为0时,可以通过变量代换将其转换为可分离变量的方程

方法如下:

如果$\frac{a_1}{a}=\frac{b_1}{b}=\lambda$,即$a_1=\lambda a$,$b_1=\lambda b$,令$u=ax+by$,两边求导可得$$\frac{du}{dx}=a+b\frac{dy}{dx}$$
原方程可以化为
\[\frac{du}{dx}=a+bf\left(\frac{u+c}{\lambda u+c_1}\right)\]
这是一个可以分离变量的方程

如果$\frac{a_1}{a}\neq\frac{b_1}{b}$,则方程组
\[\begin{cases}
ax+by+c \\
a_1x+b_1y+c
\end{cases}\]有唯一的解$x=x_0,y=y_0$,若令$\epsilon=x-x_0$,$\phi=y-y_0$,则
\[\frac{dy}{dx}=\frac{d\phi}{d\epsilon}\]
并且
\[[ax+by+c=a(\epsilon+x_0)+b(\phi+y_0)+c]=a\epsilon+b\phi\]
\[a_1x+b_1y+c_1=a_1\epsilon+b_1\phi\]
于是原方程化为
\[\frac{dy}{dx}=f\left(\frac{a\epsilon+b\phi}{a_1\epsilon+b_1\phi}\right)\]
该方程为齐次方程
\subsubsection{一阶线形微分方程}
\subsubsection{伯努利方程}
\subsection{高阶微分方程}
\subsection{线性微分方程}
形如
\[y^{(n)}+p_1(x)y^{n-1}+p_2(x)y^{(n-2)}+......+p_n(x)y=f(x)\]
的微分方程称为n阶线性微分方程

若$f(x)$恒为0,则称其为n阶线性齐次微分方程,反之则为n阶线性非齐次方程

下面引入线性相关概念:

若存在n个不全为0常数$k_1$,$k_2$......$k_n$,使得
\[k_1f_1(x)+k_2f_2(x)+......+k_nf_n(x)=0\]
在I上恒成立,则称函数组在I上线性相关,当且仅当$k_i=0$时上式成立,则方程组I上线性无关

根据定义,如果函数组线性相关可以得到
\[f_n(x)=-\frac{k_1}{k_n}f_1(x)-\frac{k_2}{k_n}f_2(x)-......-\frac{k_{n-1}}{k_n}f_{n-1}(x)\]

反之可以得到,如果其中某一个函数$f_n(x)$可以表示为其$n-1$个函数线性组合,则函数组线性相关

特别的,对于考察两个函数是否线性相关,只需看这两个函数比值是不是一个常数

\subsubsection{二阶线性微分方程}
首先看二阶线性齐次微分方程
\[y''+p(x)y'+q(x)y=0\]
此方程的解有如下性质

性质1

如果$y=y_1(x)$和$y=y_2(x)$为方程的解,则$y=y_1+y_2$也为方程的解

性质2

如果函数$y=u(x)+iv(x)$为方程的复数形式的解,则$y=u(x)$和$y=v(x)$均为方程解

根据性质可以得到定理
如果$y_1,y_2,......,y_n$为方程
\[y^{(n)}+p_1(x)y^{(n-1)}+......+p_{n-1}(x)y'+p_n(x)y\]
的n个线性无关解,则
\[y=C_1y_1+C_2y_2+...+C_ny_n\]
为方程的通解
$y_1,y_2,...,y_n$为一组基本解组

接下来看二阶线性非齐次微分方程
\[y''+p(x)y'+q(x)y=f\left(x\right)\]
其解有如下性质
对于与之对应的齐次微分方程
\[y''+p(x)y'+q(x)y=0\]
若解$y=y^.(x)$为非齐次微分方程的解,$y=y^*(x)$为齐次微分方程的解




\end{document}
