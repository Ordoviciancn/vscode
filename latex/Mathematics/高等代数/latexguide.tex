\documentclass{MathNoteCN}
\usepackage{graphicx}%可以加图片
\usepackage{physics2}
\usephysicsmodule{ab}
\usepackage{amssymb}
\usepackage{mathtools}
\title{ \LaTeX Guide}
\begin{document}. 
\section{数学公式}
段落内的叫''行内公式'', 用$...$括起来, 例如$x^2+y^2=z^2$. 

独立成行的叫''行间公式'', 用\[\]括起来,例如
\[
x^2+y^2=z^2
\]

如果需要输入上标需要给整体加花括号, 例如$x^{ln2}$. 

如果需要输入下标, 也需要加花括号, 例如$x_{i,j}$.

用frac命令可以输入分数, 例如$\frac{1}{2}$.

但是看着太小了所以通常会用dfrac命令, 例如$\dfrac{1}{2}$.%d for display

行间公式一般默认使用dfrac

有一些例外情况, 例如分式的分子和分母都很复杂, 用dfrac会导致公式过大, 用frac有些太小了, 这时候就需要用tfrac命令了, 例如
$x^{\tfrac{1}{2}}$. 

类似的,积分上下限中的分数也可以用tfrac命令, 例如
\[
\int_{0}^{\tfrac{1}{2}}x^2\, \mathrm{d}x.%标准数学文献中d不需要斜体,所以使用\mathrm{d}x
\]

\section{括号}
括号可以用left和right命令来自动调整大小, 例如
\[
\left( \dfrac{1}{2}, \dfrac{3}{4} \right) 
\]
如果只需要一侧括号, 那么在另一侧可以使用点号, 例如
\[
\left( \dfrac{1}{2}, \dfrac{3}{4} \right. 
\]

在我们调用physics2包后, 可以使用ab命令, 直接自适应括号大小, 例如
\[
\ab(\dfrac{1}{2}, \dfrac{3}{4})
\]

\section{分段函数}
分段函数可以使用cases环境来输入, 例如
\[
f(x) =
\begin{dcases}
    x^2, & x \ge 0 \\
    x-1, & x < 0. %&可以用来对齐上下行
\end{dcases}
\]

\section{更多的大写符号}

\begin{align*}
\sum_{i=1}^n i &= \dfrac{n(n+1)}{2}. \\
\prod_{i=1}^n i &= n!. \\
\bigcup_{i=1}^n A_i &= \ab\{x:\exists i\in\{1,\cdots, n\}, \, x\in A_i\}. \\
\bigcap_{i=1}^n A_i &= \ab\{x:\forall i\in\{1,\cdots, n\}, \, x\in A_i\}. \\
\end{align*}

\section{常见符号}

\begin{enumerate}
    \item 积分符号:$\int$
    \item 求和符号:$\sum$
    \item 乘积符号:$\prod$
    \item 存在符号:$\exists$
    \item forall符号: $\forall$
    \item 真符号:$\top$
    \item 假符号:$\bot$
    \item 不等于符号:$A\ne B$
    \item 小于等于符号:$A\le B$
    \item 大于等于符号:$A\ge B$
    \item 整除符号:$A\div B$
    \item 余号符号:$A\mod B$
    \item 取整符号:$\lfloor A \rfloor$
    \item 上限取整符号:$\lceil A \rceil$
    \item 开方符号:$\sqrt{A}$
    \item 最大符号:$\max(A,B)$
    \item 最小符号:$\min(A,B)$
    \item 逻辑与符号:$\land$
    \item 逻辑或符号:$\lor$
    \item 逻辑非符号:$\lnot$
    \item 波浪线符号:$\tilde{a}$
    \item ˆ符号:$\hat{a}$
    \item 下划线符号:$\underline{a}$
    \item 下划线符号:$\underbrace{11111}_{5\, \text{个}}$
\end{enumerate}

另外, 极限的写法$\lim\limits_{x\to a}f(x)=L$

\section{列表}
enumerate环境可以用来创建有序列表; 

itemize环境可以用来创建无序列表;
\end{document}